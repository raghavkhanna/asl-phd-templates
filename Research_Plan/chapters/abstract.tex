\section*{Abstract}

To feed a growing world population with the given amount of arable land, we must develop new methods of sustainable farming that increase yield while minimizing chemical inputs such as fertilizers, herbicides, and pesticides. Precision agricultural techniques seek to address this challenge by monitoring key indicators of crop health and targeting treatment only to plants or infested areas that need it. Such monitoring is currently a time consuming and expensive activity. There has been great progress on automating this activity using robots, but most existing systems have been developed to solve only specialized tasks. This lack of flexibility poses a high risk of no return on investment for farmers.
The goal of this thesis is to bridge the gap between the current and desired capabilities of autonomous unmanned aerial vehicles by developing a functional and convenient solution for precision farming. Enhancing the aerial survey capabilities of small autonomous multi-copter Unmanned Aerial Vehicles (UAVs) and providing methods to extract and analyse pertinent data gathered using onboard sensors will enable agricultural researchers and farmers to survey fields from the air and provide detailed information for decision support, paving the path towards targeted intervention on the ground, all with minimal user intervention and maximum convenience. Such a system may be adapted to a wide range of farm management activities and different crops by choosing appropriate sensors and status indicators which will be developed and tested within this thesis.